\documentclass{article}
\usepackage{amsmath}

\begin{document}

\title{Proof of the Pythagorean Theorem}
\author{}
\date{}
\maketitle

\section*{Proof}

Consider a right triangle with sides of length \(a\), \(b\), and hypotenuse \(c\).

We can construct a square with side length \(a + b\) and place four copies of the right triangle inside it, as shown below:

\[
\begin{array}{|c|c|c|c|}
\hline
& & & \\
\hline
& & & \\
\hline
& & & \\
\hline
\end{array}
\]

The area of the large square is \((a + b)^2\).

The four right triangles each have an area of \(\frac{1}{2}ab\).

The remaining area in the center is a smaller square with side length \(c\), so its area is \(c^2\).

Thus, we have:

\[
(a + b)^2 = 4 \left( \frac{1}{2} ab \right) + c^2
\]

Simplifying, we get:

\[
a^2 + 2ab + b^2 = 2ab + c^2
\]

Subtracting \(2ab\) from both sides, we obtain:

\[
a^2 + b^2 = c^2
\]

This completes the proof of the Pythagorean theorem.

\end{document}